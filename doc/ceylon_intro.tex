\subsection{How did it appear?}

Red Hat\cite{1_1} is the world's leading provider of open source software
solutions and it has initiated and sponsored\cite{1_2} the development
of the Ceylon programming language. Ceylon is first and foremost an open source
community project.

According to the language's F.A.Q\cite{1_3}, Ceylon was designed to be a
modern Java with better specification, as we can see in the excerpt:

\begin{quote}
Well, we've been designing and building frameworks and libraries for Java for
ten years, and we know its limitations intimately. And we're frustrated. The
most recent releases of Java go some distance to alleviating some problems,
but even the newest language features strain to accommodate past mistakes and
the requirement for full backward compatibility.
But much of our frustration is not even with the Java language itself. The
extremely outdated class libraries that form the Java SE SDK are riddled with
problems. Developing a great SDK is a top priority of the project.
\end{quote}

\subsection{Why was it designed and implemented?}

Ceylon has a lot of interesting features but, we can start with this
definition on the homepage\cite{1_4}: \textit{"Ceylon is a language for writing
large programs in teams"}. The language is deeply influenced by Java and it was
designed and implemented by people who were hugely involved with Java. People
like Gavin King, the lead of the Ceylon project in Red Hat and also the creator
of Hibernate\cite{1_5} (the most popular Object Relational Mapper for Java) and
other projects in the Oracle's platform.

Modularity is in the very core of the Ceylon language. When a Ceylon's code is
compiled, it produces module archives. In fact is very similar to Java's
modules. It generates a *.car file, with *.class files zipped on it. Just
like Java's *.jar. You'll never be exposed to unpacked *.class files in
Ceylon. So here we got 2 conclusions. The first one is that the compiler
really forces the modularity and facilitates the distribution of the generated
code. The second one is that the Ceylon compiled code runs on the Java Virtual
Machine (JVM). This second fact is very important because by running on the JVM,
Ceylon's code is fully interoperable with Java, the Java SDK and its libraries.
In fact, this interoperability is a major priority of the project.

This is an example of the use of the Java's HashMap data structure on a Ceylon
program:

\begin{lstlisting}[label=cujhm,caption=Ceylon using Java HashMap]
import java.util { HashMap }

value javaHashMap = HashMap<String,Integer>();
javaHashMap.put("zero", 0);
javaHashMap.put("one", 1);
javaHashMap.put("two", 2);
print(javaHashMap.values());
\end{lstlisting}

Ceylon runs on the JavaScript virtual machine too. In fact the Ceylon's compiler
can generate JavaScript code if told to do so. It generates modular JavaScript
in the CommonJS modules format. We'll see more on JavaScript generation in the
future.

\begin{lstlisting}[label=cjs,caption=Ceylon JavaScript]
dynamic {
    dynamic req = XMLHttpRequest();
    req.onreadystatechange = void () {
        if (req.readyState==4) {
            document.getElementById("greeting")
                    .innerHTML = req.status==200
                            then req.responseText
                            else "error";
        }
    };
    req.open("GET", "/sayHello", true);
    req.send();
}
\end{lstlisting}

Ceylon's project always tells how important is the toolset for a complete and
successful programing project so it ships with a really great Integrated
Development Environment (IDE) built on a Eclipse plug-in. We'll see more
detailed information on the {install and run} section below.

\subsection{How can we use it, e.g., install and run?}

By running in the JVM, the Ceylon's compiler got the advantage to run
on every operating system which has a Java Virtual Machine implementation. It
runs both on Java 7 and Java 8 (prior versions are not supported). So before
installing the Ceylon package, be sure to have the correct Java version
installed. On this work, I'm using the 1.1 version of the language, released on
09 October 2014.

For installation on a Mac, you can use the Homebrew installer:

\begin{verbatim}
$ brew update
$ brew install ceylon
\end{verbatim}

There are packages for both Debian and Fedora/Red Hat GNU/Linux flavors on the
project's download page\cite{1_10}.

For a platform agnostic installation, download the zip archive
(\url{http://ceylon-lang.org/download/dist/1_1_0}),
unzip on your system's prefered folder and add the \textbf{/ceylon-1.1.0/bin}
folder to the path of your system. In a unix-like system you can do that by
adding the line below on the \textbf{~/.bashrc} file of you user's directory:

\begin{verbatim}
$ export PATH="/path/to/ceylon-1.1.0/bin:$PATH"
\end{verbatim}

Now let's check if the installation works. By forcing the modularity, Ceylon
implies some conventions on the compilation process. At first the code we'll
write must be on a \textbf{"source"} directory. So place the following code on a
file called \textbf{"./source/hello.ceylon"}:

\begin{lstlisting}[label=hello,caption=Ceylon Hello World]
shared void hello() {
  print("Hello Ceylon!");
}
\end{lstlisting}

Outside the \textbf{source's} folder, run the command:

\begin{verbatim}
$ ceylon compile source/hello.ceylon
\end{verbatim}

Checkout the \textbf{module} folder created on the same level as the
\textbf{source} folder. Check it's content for the \textit{*.car} file and some
others. To run the compiled source, run the following command:

\begin{verbatim}
$ ceylon run --run hello default
\end{verbatim}

\subsection{Simple programs}

I will demonstrate some features of the language by writing some simple programs
and analyzing some points right after it.

\subsubsection{The treelike structure, simple classes and visibility}

I'll use this example to illustrate the treelike structure that Ceylon has.
Ceylon's named argument lists provide an elegant way for initializing objects
and collections. The goal of this facility is to replace the use of XML for
expressing hierarchical structures such as documents, user interfaces,
configuration and serialized data.

I'll use this notation to create a binary search tree and make a depth-first
in-order traversal.

\begin{lstlisting}[label=ctls,caption=Ceylon treelike structure]
shared class Node(val, left = null, right = null) {
	shared Integer val;
	shared Node? left;
	shared Node? right;
}

shared void caminhamento_central(Node b) {
	if(exists next = b.left) {
		caminhamento_central(next);
	}
	process.write(b.val.string + " ");
	if(exists next = b.right) {
		caminhamento_central(next);
	}
}

shared void run() {
	Node arvore = Node {
		val = 1;
		left = Node {
			val = 2;
			left = Node {
				val = 4;
				left = Node { val = 8; };
				right = Node { val = 9; };
			};
			right = Node {
				val = 5;
				left = Node { val = 10; };
				right = Node { val = 11; };
			};
		};
		right = Node {
			val = 3;
			left = Node {
				val = 6;
				left = Node { val = 12; };
				right = Node { val = 13; };
			};
			right = Node {
				val = 7;
				left = Node { val = 17; };
				right = Node { val = 15; };
			};
		};
	};

	caminhamento_central(arvore);
	print("");
}
\end{lstlisting}

The result of this program must be:

\begin{verbatim}
8 4 9 2 10 5 11 1 12 6 13 3 17 7 15
\end{verbatim}


Look at the word \textit{shared} on some points of the example. This \textit{shared}
annotation marks a declaration as being visible outside the scope in which it is
defined, or a package as being visible outside the module to which it
belongs\cite{1_6}.

It's all about visibility\cite{1_7}. Classes, interfaces, functions, values,
aliases, and type parameters have names. Occurrence of a name in code implies a
hard dependency from the code in which the name occurs to the schema of the
named declaration. We say that a class, interface, value, function, alias, or
type parameter is visible to a certain program element if its name may occur in
the  code that belongs to that program element. The visibility of a declaration
depends upon where it occurs, and upon whether it is annotated \textit{shared}.

\subsubsection{Random numbers and Java interoperability}

Ceylon's SDK is under constant development\cite{1_8} and there aren't a random
numbers implementation yet. But Ceylon can use the Java SDK, so I'll use the
Java random numbers interface to work on a Ceylon's simple example.

\begin{lstlisting}[label=w1,caption=Random number from Java]
// Import from Java SDK
import java.util { Random }

shared Integer getRandomInteger(Integer a, Integer b, Random r) {
	value range = b - a + 1;
	value fraction = (range * r.nextDouble());
	return (fraction + a).integer;
}

shared void run() {
    Random random = Random();
    for(number in 1..100) {
        print(getRandomInteger(1, 100, random));
    }
}
\end{lstlisting}

To use the Java SDK (and other libraries from outside the language's SDK) you
must use the module structure that is conventioned in the language. I'll use
this example to show it.

Save the code above on a file called \textbf{random.ceylon} inside of the
following structure of directories: \textbf{./source/example/random/}. This
structure corresponds to the namespace of the module we are creating. It's just
like Java always did.

Inside this same folder, create the file \textbf{module.ceylon}. It is the file
that will specify the dependencies our module have with external resources (like
Java libraries) and formalize it's namespace. The content of this file must be:

\begin{lstlisting}
module example.random "1.0.0" {
    import java.base "8";
}
\end{lstlisting}

The \textit{java.base} module in JDK has Java base packages such as
\textit{java.lang}, \textit{java.util}, \textit{java.io}, \textit{java.net},
\textit {java.text}, NIO and security\cite{1_9}. All done, inside of the
project's root, run the command:

\begin{verbatim}
$ ceylon compile
\end{verbatim}

The module will be compiled with the correct dependencies and the files will be
generated in the correspondent directory of the module (in our example, it
would be $./modules/example/random/1.0.0/$). Then we can run the code
with the command:

\begin{verbatim}
$ ceylon run example.random
\end{verbatim}

The program will execute calling the method \textit{run} (like the \textit{main}
method in Java) inside the package \textit{example.random}.
