\documentclass{abnt}
\usepackage[utf8]{inputenc}
\usepackage[num]{abntcite}
\usepackage{graphicx}
\usepackage{url}
\usepackage{caption}
\usepackage{listings}
\usepackage[brazil]{babel}
\addto\captionsbrazil{%
\def\bibname{References}%
\def\abstract{Abstract}%
}

\lstset{
basicstyle=\small\sffamily,
numbers=left,
numberstyle=\tiny,
frame=tb,
columns=fullflexible,
showstringspaces=false
}

\graphicspath{{imagens/}}

\begin{document}
\autor{André Taiar Marinho Oliveira}

\titulo{A comparative study of concurrency models in Elixir and Rust}

\orientador[Orientador:\\]{Prof. Dr. Fernando Magno Quintão Pereira}

\comentario{Apresentado como requisito da disciplina de Monografia em Sistemas de Informação do DCC/UFMG}

\instituicao{Universidade Federal de Minas Gerais \par Instituto de Ciências
Exatas \par Departamento de Ciência da Computação}

\local{Belo Horizonte} \data{2016/2}

\capa
\folhaderosto

\begin{resumo}
There are hundred of programming languages out there. Which one should we use?
Which help do we have to choose well? How do they compare to each other? This
document is an attempt to provide some answers to these questions. Naturally, it
would not be possible to provide complete answers: as I mentioned, there are too
many programming languages. Nevertheless, we chose four languages with a
potential to grow in importance in the coming years. These programming languages
are Ceylon, Dart, Elixir and Rust. During this project, we shall be
talking about each one of them. These discussions will be in breath, not in
depth. Their goal is to provide the reader with the minimum of information
necessary to compare them, and who knows, to lure one or other interested person
in learning them in a greater level of details. In any case, we hope to
contribute a bit to the popularization of these programming languages, which -
likely - will be paramount to the development of computer science in the next
ten years.\\

\textbf{Keywords}: programming languages, concurrency, elixir, rust.
\end{resumo}

\sumario
\listoffigures

\chapter{Elixir}
\section{Elixir language introduction}

\subsection{How did it appear?}

Elixir \cite{5_1} is a programming language
designed to build scalable and maintainable systems. It first appeared in may
2012 after 1 year of \textit{underground development} and at first it was a solo
project from its creator and maintainer José
Valim \cite{5_2}. In the begining of 2012,
Plataformatec \cite{5_3}, a brazilian tech company
where José Valim works (and is one of the owners) starts to sponsor and support
the project and then the language got a real improvement\cite{brown1998chemoinformatics}.

Even before the first stable version was launched, Elixir attract many
contributors and enthusiasts. It counts 6 written books about programming in
Elixir and other recipes \cite{5_4}, 3
international conferences \cite{5_5} and 300+
open source contributors \cite{5_6}. Besides having
its own community, Elixir is also attracts members of the Ruby and Erlang
communities too.

The version 1.0 of the language \cite{5_7} was
released in September 10 in the year of 2014.

\subsection{Why was it designed and implemented?}

Back in 2010, José Valim was a very active member of the Ruby
Language \cite{5_8} community. He worked on the
core team of a major Ruby open source project called Ruby on
Rails \cite{5_9} (a framework to build web
applications). One of his contributions for this tool was to make the Ruby on
Rails framework **thread safe**. At first, Ruby on Rails applications doesn't
scaled horizontally. At that time, there was no advantage on running a Rails
application on multi-core machines. Making the framework thread safe means that
the application written in Ruby on Rails when deployed in multi-core machines
would run efficiently and use all processing power of that environment and in
the right way (there will exist no problems with concurrency). Valim finds out
that, on that time, there weren't good tools on Ruby ecosystem to work with
concurrency and he started do research about this topics trying to find in other
languages and platforms the best ways to solve the problem they had in Rails.

During his explorations, Valim read the awesome \textbf{7 languages in 7 weeks}
\cite{5_10} where he meets the Erlang \cite{5_11} programming language. Erlang
is a functional programming language and it has a virtual machine. It was
developed by Ericsson \cite{5_12} and open sourced in 1998. It was designed for
distributed and fault tolerant applications that runs in real time in a
uninterrupted environment.

Very interested in the platform, he started to look deeper in the Erlang
ecossystem writing programs and projects in Erlang. He loved the things the
language has but misses things that he thought to be very important like good
metaprogramming and polimorphysm. Despite these misses, the Erlang Virtual
Machine was really awesome and Erlang was very good in concurrency.

At that time, there was only one language that had all the things Valim wishes
to have in a language and it was Clojure \cite{5_13}.
Clojure is a dynamic language (no statyc typing), very extensible and with great
focus on concurrency. Ruby was a dynamic and extensible language but was not
good in concurrency scenarios like we saw. Go \cite{5_14}
has a focus in concurrency but is static typed and not a very extensible
language. Clojure has all he wants but is a JVM language (language that runs in
the Java Virtual Machine) and he thinks that was a good idea to have other
language option on this niche.

\subsection{How can we use it, e.g., install and run?}

The only prerequisite to run Elixir programs is Erlang 17.0 or later. So, before
install the Elixir you must be sure to have Erlang installed.

\subsubsection{Linux}

There are easy install setups for all major Linux distributions. Most of them
installs the Erlang Virtual Machine too, so yout don't have do care with it. You
can install on Ubuntu based distros with the commands:

\begin{verbatim}
$ wget https://packages.erlang-solutions.com/erlang-solutions_1.0_all.deb && sudo dpkg -i erlang-solutions_1.0_all.deb
$ sudo apt-get update
$ sudo apt-get install elixir
\end{verbatim}

on Red Hat distros with:

\begin{verbatim}
$ yum install elixir
\end{verbatim}

and on Arch Linux with the command:

\begin{verbatim}
$ pacman -S elixir
\end{verbatim}

\subsubsection{Mac OS}

On Mac OS you can install using Homebrew:

\begin{verbatim}
$ brew update
$ brew install elixir
\end{verbatim}

\subsubsection{Windows}

There are precompilled packages for Windows you can download and install on the
Elixir download session at the website. You can also use the Chocolatey tool
and install with:

\begin{verbatim}
cinst elixir
\end{verbatim}

\subsubsection{Testing the installation}

After installing, let's check if we installed right and we can run Elixir
programs in our environment. Elixir code can be scripted and interpreted or
compiled into bytecode that runs on the Erlang VM. Let's checkt the interpreter.

Write and save the following code to the $hello.exs$ file:

\begin{lstlisting}[label=ehw,caption=Elixir Hello World]
IO.puts "Hello, Elixir!"
\end{lstlisting}

and then, run the commands:

\begin{verbatim}
$ elixir hello.exs
\end{verbatim}

and you'll have the $Hello, Elixir!$ sentence on your screen.

For the compilation to bytecodes, the Elixir program must be inside a module.
Write the $hello.ex$ file with the following contents:

\begin{lstlisting}[label=emhw,caption=Elixir Hello World inside a module]
defmodule Hello do
  def say do
    IO.puts "Hello, Elixir!"
  end
end
\end{lstlisting}

and compile with the command:

\begin{verbatim}
$ elixirc hello.ex
\end{verbatim}

Check that the $Elixir.Hello.beam$ file is created with bytecode in the same
directory. Let's check if the module works using the Elixir interactive
environment. Run the program in the command line in this same directory:

\begin{verbatim}
$ iex
Erlang/OTP 18 [erts-7.1] [source] [smp:4:4] [async-threads:10] [kernel-poll:false]

Interactive Elixir (1.1.1) - press Ctrl+C to exit (type h() ENTER for help)
iex(1)>
\end{verbatim}

In the interactive execution environment, type the code:

\begin{verbatim}
iex(1)> Hello.say()
Hello, Elixir!
:ok
\end{verbatim}

and check that it loads the module and executes the function we defined.

\subsection{Simple programs}

Elixir is a functional language (the only functional-only language I learned on
this work). Here are its basic types:

\begin{verbatim}
iex> 1          # integer
iex> 0x1F       # integer
iex> 1.0        # float
iex> true       # boolean
iex> :atom      # atom / symbol
iex> "elixir"   # string
iex> [1, 2, 3]  # list
iex> {1, 2, 3}  # tuple
\end{verbatim}

There are some common features in functional languages that we'll
look on the next examples.

\subsubsection{Anonymous functions}

In the interactive execution environment, let's see some things about anonymous
functions:

\begin{verbatim}
$ iex
Erlang/OTP 18 [erts-7.1] [source] [smp:4:4] [async-threads:10] [kernel-poll:false]

Interactive Elixir (1.1.1) - press Ctrl+C to exit (type h() ENTER for help)
iex(1)> add = fn a, b -> a + b end
#Function<12.54118792/2 in :erl_eval.expr/5>
iex(2)> is_function(add)
true
iex(3)> add
#Function<12.54118792/2 in :erl_eval.expr/5>
iex(4)> add.(2, 4)
6
iex(5)> add_ten = fn a -> add.(a, 10) end
#Function<6.54118792/1 in :erl_eval.expr/5>
iex(6)> add_ten.(11)
21
iex(7)> x = add_ten.(0)
10
iex(8)> x
10
iex(9)> (fn -> x = 42 end).()
42
iex(10)> x
10
\end{verbatim}

Functions are delimited by the keywords **fn** and **end**. They can be assigned
to variables and passed as function parameters just as other types in the
language. In the example, we assing a function that add two variables to a
variable called $add$. To test it, we invoke the anonymous function with $add.$
(we need that point after the variable name to call the anonymous functions) and
then the parameters.

Anonymous functions are clojures and they can access variables that are in scope
when the function is defined, like in the example when we create the $add_ten$
variable. The variable assigned inside a function doesn't affects the outer
scope, so we check it on the last part of the execution.

\subsubsection{Lists, tuples and immutability}

Here are some operations with lists:

\begin{verbatim}
$ iex
Erlang/OTP 18 [erts-7.1] [source] [smp:4:4] [async-threads:10] [kernel-poll:false]

Interactive Elixir (1.1.1) - press Ctrl+C to exit (type h() ENTER for help)
iex(1)> music = [:a, :b, :c] ++ [1, 2, 3]
[:a, :b, :c, 1, 2, 3]
iex(2)> music -- [:c, 3]
[:a, :b, 1, 2]
iex(3)> hd(music)
:a
iex(4)> tl(music)
[:b, :c, 1, 2, 3]
\end{verbatim}

There are the concatenation operator **++** and the subtractor **--** operators
that concatenates and subtract lists and there are the **hd** and **tl**
functions that returns the head and the tail of the list.

\begin{verbatim}
$ iex
Erlang/OTP 18 [erts-7.1] [source] [smp:4:4] [async-threads:10] [kernel-poll:false]

Interactive Elixir (1.1.1) - press Ctrl+C to exit (type h() ENTER for help)
iex(1)> tuple = {:ok, "hello"}
{:ok, "hello"}
iex(2)> put_elem(tuple, 1, "world")
{:ok, "world"}
iex(3)> elem(tuple, 0)
:ok
iex(4)> tuple_size(tuple)
2
iex(5)> elem(tuple, 1)
"hello"
\end{verbatim}

We create a tuple using curly brackets. We can set a value for a determined
tuple position, return a position and get a tuple size. Note in the last command
that the second element of the variable tuple remains the same. The original
tuple stored in the tuple variable was not modified because Elixir data types
are immutable. By being immutable, Elixir code is easier to reason about as you
never need to worry if a particular code is mutating your data structure in
place.

Lists in Elixir are linked. This means that each element in a list holds its
value and points to the following element until the end of the list is reached.
Accessing the length of a list is a linear operation: we need to traverse the
whole list in order to figure out its size. Updating a list is fast as long as
we are prepending elements.

Tuples are stored contiguously in memory. This means getting the tuple size or
accessing an element by index is fast. However, updating or adding elements to
tuples is expensive because it requires copying the whole tuple in memory. Those
performance characteristics dictate the usage of those data structures.

\subsubsection{Pattern matching}

Pattern matching allows to easily destructure data types such as tuples and
lists. It is one of the foundations of recursion in Elixir. Here are some
examples matching lists and tuples.

\begin{verbatim}
$ iex
Erlang/OTP 18 [erts-7.1] [source] [smp:4:4] [async-threads:10] [kernel-poll:false]

Interactive Elixir (1.1.1) - press Ctrl+C to exit (type h() ENTER for help)
iex(1)> {a, b, c} = {:hello, "world", 42}
{:hello, "world", 42}
iex(2)> b
"world"
iex(3)> [a, b, c] = [1, 2, 3]
[1, 2, 3]
iex(4)> b
2
iex(5)> list = [1, 2, 3]
[1, 2, 3]
iex(6)> [0|list]
[0, 1, 2, 3]
iex(7)> [head | tail] = [1, 2, 3]
[1, 2, 3]
iex(8)> head
1
iex(9)> tail
[2, 3]
\end{verbatim}

\subsubsection{Modules}

Elixir groups several functions into modules. In order to create our own modules
we use the **defmodule** macro. We use the **def** macro to define functions in
that module.

\begin{lstlisting}[label=esum,caption=Elixir Module example]
defmodule Math do
  def sum(a, b) do
    a + b
  end
end

Math.sum(1, 2)
\end{lstlisting}

Inside a module, we can define functions with **def** and private functions with
**defp**. A function defined with **def** can be invoked from other modules
while a private function can only be invoked locally.

\begin{lstlisting}[label=econc,caption=Elixir pattern matching and function visibility]
defmodule Concato do

  def concat(a, b) do
    do_concat(a, b)
  end

  defp do_concat([h|t], [a|b]) do
    [h|t] ++ [a|b]
  end

  defp do_concat(a, b) do
    a <> b
  end

end

IO.puts Concato.concat("Merry", "Christmas") #=> MerryChristmas
IO.inspect Concato.concat([:a, :b, :c], [1, 2, 3]) #=> [:a, :b, :c, 1, 2, 3]
\end{lstlisting}

The last example show the usage of a module with two private functions and a
public function. The public function is used for interface with the module and
it calls the possible 2 other functions using pattern matching. When the
arguments are two lists, it calls a function that concatenate lists and when
they are strings, the function that concatenates strings is called.

\subsubsection{Recursion}

Due to immutability, loops in Elixir (as in any functional programming language)
are written differently from imperative languages. Functional languages rely on
recursion: a function is called recursively until a condition is reached that
stops the recursive action from continuing.

\begin{lstlisting}[label=errrr,caption=Elixir recursion]
defmodule Recursion do

  def print_multiple_times(msg, n) when n <= 1 do
    IO.puts msg
  end

  def print_multiple_times(msg, n) do
    IO.puts msg
    print_multiple_times(msg, n - 1)
  end

end

Recursion.print_multiple_times("Gol!", 7)
\end{lstlisting}

Let's use recursion to make some math operations. The example above has three
**public** functions: the first one return the sum of all numbers in the list
and the second return the list with all elements multiplied by two.

\begin{lstlisting}[label=elists,caption=Elixir recursion with lists]
defmodule Math do

  def sum_list([h|t]) do
    sum_list([h|t], 0)
  end

  defp sum_list([head|tail], accumulator) do
    sum_list(tail, head + accumulator)
  end

  defp sum_list([], accumulator) do
    accumulator
  end

  def double_each([head|tail]) do
    [head * 2|double_each(tail)]
  end

  def double_each([]) do
    []
  end

end

IO.puts Math.sum_list([2,3,5]) #=> 10
IO.inspect Math.double_each([2,3,5]) #=> [4, 6, 10]
\end{lstlisting}

Elixir has a Enum \cite{5_15} module that
provides a set of algorithms that enumerate over collections. This module has a
**reduce** and a **map** functions that we'll use to rewrite our last example:

\begin{lstlisting}[label=eenum,caption=Elixir lists with Enum]
defmodule Mathenum do

  def sum_list([h|t]) do
    Enum.reduce([h|t], 0, fn(x, acc) -> x + acc end)
  end

  def double_each([h|t]) do
    Enum.map([h|t], fn(x) -> x * 2 end)
  end

end

IO.puts Mathenum.sum_list([2,3,5]) #=> 10
IO.inspect Mathenum.double_each([2,3,5]) #=> [4, 6, 10]
\end{lstlisting}

\section{Elixir language usage}
On this usage example, we'll use most of the concepts on the last article to 
build a more complex program that explores the concurrent features of Elixir.
Starting with a Fibonacci function we'll increment the program touching concepts
that were not in the first article like pipes, processes, communication between 
processes and tail call optimization.

\subsection{Fibonacci}

Let's begin with a canonical starter recursive Fibonacci implementation with 
Elixir. Here is the code:

\begin{lstlisting}[label=efib1,caption=Classic Fibonacci]
defmodule Fib do

  def fib(0) do 0 end
  def fib(1) do 1 end
  def fib(n) do fib(n - 1) + fib(n - 2) end

end

IO.puts Fib.fib(0)
IO.puts Fib.fib(1)
IO.puts Fib.fib(2)
IO.puts Fib.fib(3)
IO.puts Fib.fib(4)
IO.puts Fib.fib(5)
IO.puts Fib.fib(6)
IO.puts Fib.fib(7)
IO.puts Fib.fib(8)
\end{lstlisting}

Nothing new here, just right to the point. Just write the problem's definition 
and we got the right result:

\begin{verbatim}
  
\end{verbatim}

\begin{verbatim}
$ elixir fib.ex 
0
1
1
2
3
5
8
13
21
\end{verbatim}

\subsection{Many Fibonacci numbers}

Let's improve our last code to generate many numbers passing them as a list to 
the function \cite{6_2}. Here is the code:

\begin{lstlisting}[label=emf,caption=Many Fibonacci numbers]
defmodule Fib do

  def run(list) do
    list
    |> Enum.map(&(fib(&1)))
    |> inspect
    |> IO.puts
  end

  def fib(0) do 0 end
  def fib(1) do 1 end
  def fib(n) do fib(n - 1) + fib(n - 2) end

end

Fib.run([0,1,2,3,4,5,6,7,8])
\end{lstlisting}

The $|>$ symbol used in the snippet above is the pipe operator: it simply 
takes the output from the expression on its left side and passes it as the first
argument to the function call on its right side. It’s similar to the Unix **|** 
operator. Its purpose is to highlight the flow of data being transformed by a 
series of functions. The $run$ function without the pipe operator would be:

\begin{lstlisting}[label=dartMap,caption=Dart dynamic type]
def run(list) do
  IO.puts(inspect(Enum.map(list, fn(x) -> fib(x) end )))
end
\end{lstlisting}

The $Enum.map$ function calls the function on the second parameter for each 
element in the list passed as the first parameter returning a new list with the 
transformed elements. The $inspect$ function returns a String with the list
\"pretty printed\" (written in a human readable format) and we print the list with
the $IO.puts$ function. The result of the execution, for whatever $run$ 
implementations is:

\begin{verbatim}
$ elixir fib.ex 
[0, 1, 1, 2, 3, 5, 8, 13, 21]
\end{verbatim}


\subsection{Parallel Fibonacci numbers}

In our last example, we run the function that calculates each Fibonacci's number
sequentially. The execution waits each of the numbers to be calculated before 
calculate the next number. Running this program for some bigger numbers would be
slow. Let's measure the time to calculate the Fibonacci numbers from 30 to 40:

\begin{verbatim}
$ time elixir fib.ex 
[832040, 1346269, 2178309, 3524578, 5702887, 9227465, 14930352, 24157817, 39088169, 63245986, 102334155]

real  1m2.186s
user  1m1.664s
sys 0m0.684s
\end{verbatim}

Now I'll rewrite the program to calculate these numbers using data parallelism.
Our code will distribute the number calculation across different parallel 
computer cores possibly making the execution faster. Here is the code:

\begin{lstlisting}[label=epf,caption=Parallel processing for Fibonacci in Elixir]
defmodule Fib do

  def run(list) do
    list
    |> Enum.with_index
    |> Enum.map fn(ni) -> spawn_run(self, ni) end
    receive_fibs(length(list), [])
  end

  defp receive_fibs(lns, result) do
    receive do
      fib ->
        result = [fib | result]
        if lns == 1 do
          IO.puts(print_fibs(result))
        else
          receive_fibs(lns - 1, result)
        end
    end
  end

  defp print_fibs(fibs) do
    fibs
    |> Enum.sort(fn({_, a}, {_, b}) -> a < b end)
    |> Enum.map(fn({f, _}) -> f end)
    |> inspect
  end

  def spawn_run(pid, ni) do
    spawn __MODULE__, :send_run, [pid, ni]
  end

  def send_run(pid, {n, i}) do
    send pid, {fib(n), i}
  end

  def fib(0) do 0 end
  def fib(1) do 1 end
  def fib(n) do fib(n - 1) + fib(n - 2) end
end

Fib.run([30,31,32,33,34,35,36,37,38,39,40])
\end{lstlisting}

Our already existent $run$ function was modified. The $Enum.with_index$ 
method returns the collection with each element wrapped in a tuple alongside its
index. This way we can identify this number later to return them in the correct
order. The $Enum.map$ function that used to call the $fib$ function now 
calls a $spawn_run$ function. At the end of the $run$ function it calls a 
$receive_fibs$ function.

The $spawn_run$ function just spawn off new processes \cite{6_1}
of execution. The Elixir's $spawn$ function takes a function which it will 
execute in another process. It returns a PID (process identifier). The spawned 
process will execute the given function and exit after the function is done. The
program uses the $send$ and $receive$ functions to comunicate between 
different proccesses. The $send$ method receives a PID and a message and sends
the message to the proccess with the referred PID. The $receive$ function has 
one parameter that must be a function that will receive the potentially message 
sent. 

So in our code, the $spawn_run$ is called (for each one of the numbers we want
the fibonacci number) and it calls a $send_run$ function. The $send_run$ 
function sends back a result of our already known $fib$ functions. The receive
function is used inside the $receive_fibs$ function. $receive_fibs$ works as
a synchronization function that awaits the execution of all parallel $fib$ 
calls, adding them to a list and returning this list when there is no more 
proccesses running.

When all the execution is over, the $receive$ method calls the $print_fibs$
function. This function orders the calculated number by its index (mentioned on 
the $run$ function). This is necessary because the result of the parallel 
and asynchronous execution times are not deterministic and the order it returns 
the numbers is unknown. At last, the tuples with indexes are removed and just 
the result of the Fibonacci numbers are printed on the screen.

Now, this is the execution time of the parallel version of the program:

\begin{verbatim}
$ time elixir fib.ex 
[832040, 1346269, 2178309, 3524578, 5702887, 9227465, 14930352, 24157817, 39088169, 63245986, 102334155]

real  0m33.141s
user  1m22.564s
sys 0m1.208s
\end{verbatim}

The real execution time of the parallel program is almost the half of the 
execution time for the sequential version of the code. This time would become 
even better if the machine that executes the program has a processor with a 
bigger number of execution cores.

\subsection{Tail call optimization}

Tail-call optimization is where you are able to avoid allocating a new stack 
frame for a function because the calling function will simply return the value 
that it gets from the called function.

The parallel portion of the implementations just remains the same. The 
modification is on the $fib$ functions that now has a tail call, avoiding so 
many recursion calls. Here is the code:

\begin{lstlisting}[label=dartMap,caption=Parallel Fibonacci with tail call optimization]
defmodule Fib do

  def run(list) do
    list
    |> Enum.with_index
    |> Enum.map fn(ni) -> spawn_run(self, ni) end
    receive_fibs(length(list), [])
  end

  defp receive_fibs(lns, result) do
    receive do
      fib ->
        result = [fib | result]
        if lns == 1 do
          IO.puts(print_fibs(result))
        else
          receive_fibs(lns - 1, result)
        end
    end
  end

  defp print_fibs(fibs) do
    fibs
    |> Enum.sort(fn({_, a}, {_, b}) -> a < b end)
    |> Enum.map(fn({f, _}) -> f end)
    |> inspect
  end

  def spawn_run(pid, ni) do
    spawn __MODULE__, :send_run, [pid, ni]
  end

  def send_run(pid, {n, i}) do
    send pid, {fib(n), i}
  end

  def fib(n) do fib(n, 1, 0) end
  def fib(0, _, _) do 0 end
  def fib(1, a, b) do a + b end
  def fib(n, a, b) do fib(n - 1, b, a + b) end

end

Fib.run([30,31,32,33,34,35,36,37,38,39,40])
\end{lstlisting}

It might be more illustrative if shown in a non-functional language, like Python:

\begin{lstlisting}[label=ppppp,caption=Python tail call optimization for Fibonacci]
def fib(i, current = 0, next = 1):
  if i == 0:
    return current
  else:
    return fib(i - 1, next, current + next)
\end{lstlisting}

We can see that now, the real execution time of our program is very smaller than
our two last versions:

\begin{verbatim}
$ time elixir fib.ex 
[832040, 1346269, 2178309, 3524578, 5702887, 9227465, 14930352, 24157817, 39088169, 63245986, 102334155]

real  0m2.321s
user  0m2.152s
sys 0m0.320s
\end{verbatim}


\chapter{Rust}
\section{Rust language introduction}
\subsection{How did it appear?}

Rust  \cite{7_1} is a general-purpose,
multi-paradigm, compiled, systems programming
language. Nowadays, the Rust project is maintained and developed by Mozilla  \cite{7_2}
Research but it starts as a personal project by Mozilla employee Graydon Hoare
 \cite{7_3}  \cite{7_4}.

Graydon starts working on Rust in 2006. He was already a engineer at Mozilla
working on compilers and tools for other languages. He started sketching Rust
thinking in a lot of obvious good ideas, known and loved in other languages,
that haven't made it into widely-used systems languages, or are deployed in
languages that have very poor (unsafe, concurrency-hostile) memory models.

After working long time as a hobby project, he decided to show one such
prototype he'd been working on his spare time to his manager. Mozilla took an
interest and set up a team to work on this language as a component of
longer-term project to rebuild their browser stack around safer, more
concurrent, easier technologies than C++. That larger project is called \"servo\".
Mozilla is funding Rust development because of that.

Mozilla began sponsoring the project in 2009 and announced it in 2010. The
first Rust compiler, that was written in OCaml 
 \cite{7_5}, was dropped and rewritten in a
Rust version that successfully compiled itself in 2011. It is called $rustc$
and uses LLVM  \cite{7_6} as it's back end. The
version 1.0  \cite{7_7} was released in 15 May
2015.

\subsection{Why was it designed and implemented?}

System's programming languages became widely used on the last 50 years since
people started using high-level languages to write operating systems.
Nevertheless, there were always two problems constantly present in this niche:

\begin{itemize}
    \item It's difficult to write secure code (due to the way that languages like C and
          C++ manages memory);
    \item It's hard (but essential) to write multi-threaded concurrent code due to the
          low support the languages of this niche offers to programmers to deal
          with the bugs that writing parallel programs would offer  \cite{7_8}.
\end{itemize}

These are the problems that Rust was made to address. As it's own creator says:

\begin{quotation}
Our target audience is frustrated C++ developers. 
\end{quotation}

on which he includes himself. The performance of safe code (which is one of
things that Rust aims) is expected to be slower than C++, however, the Rust
program's performance is comparable to C++ code that manually implements safer
operations  \cite{7_9}. 

Rust is a compiled, multi-paradigm programming language. It supports
pure-functional, actor-based concurrency, imperative-procedural and
object-oriented programming styles.

\subsection{How can we use it, e.g., install and run?}

\subsubsection{Linux and Mac OS}

Rust has a simple installation proccess on Linux and Mac OS systems. Run the
following line in the command terminal:

\begin{verbatim}
$ curl -sSf https://static.rust-lang.org/rustup.sh | sh
\end{verbatim}

It will download the files, ask for $sudo$ passwords and setup everything
automatically.

\subsubsection{Windows}

There are binaries versions with installers for Windows users at the download
page  \cite{7_10}.

\subsubsection{Testing the installation}

Rust default toolset ships with a build tool called $cargo$
 \cite{7_11}. It is a tool to help manage the
Rust projects. Cargo manages three things: building your code, downloading the
dependencies your code needs, and building those dependencies.

To start a new project with cargo, run the command:

\begin{verbatim}
$ cargo new hello --bin
\end{verbatim}

It will generate the dicrectory called $hello$ with some files on it. On the
directory root it creates a $Cargo.toml$ file with some basic information about
the project and a $src$ directory with a $main.rs$ file on it. This is the
content of the main file:

\begin{lstlisting}[label=rhw,caption=Rust Hello World]
fn main() {
    println!("Hello, world!");
}
\end{lstlisting}

To compile and run the project, in the project's root directory, run the
commands:

\begin{verbatim}
$ cargo build
   Compiling hello v0.1.0 (file:///home/taiar/dev/mono1/rust/hello)
$ cargo run
     Running `target/debug/hello`
Hello, world!
\end{verbatim}

and the hello message just show up. The binary file is located at
$target/debug/hello$ and can be executed directly from the command line
without the cargo tool too.

\begin{verbatim}
$ ./target/debug/hello
Hello, world!
\end{verbatim}

\subsection{Simple programs}

\subsubsection{Primitive types and mutability}

Rust provides access to a wide variety of primitives:

\begin{itemize}
    \item signed integers: $i8$, $i16$, $i32$, $i64$ and $isize$ (pointer size);
    \item unsigned integers: $u8$, $u16$, $u32$, $u64$ and $usize$ (pointer size);
    \item floating point: $f32$, $f64$;
    \item char Unicode scalar values like $'a'$, $\alpha$ and $\infty$ (4 bytes each);
    \item bool either $true$ or $false$;
    \item arrays like $[1, 2, 3]$;
    \item tuples like $(1, true)$.
\end{itemize}

Here are some usage:

\begin{lstlisting}[label=rst,caption=Rust types usage]
fn main() {
    // Variables can be type annotated.
    let logical: bool = true;

    let a_float: f64 = 1.0;  // Regular annotation
    let an_integer   = 5i32; // Suffix annotation

    // Or a default will be used.
    let default_float   = 3.0; // `f64`
    let default_integer = 7;   // `i32`

    let mut mutable = 12; // Mutable `i32`.
}
\end{lstlisting}

Variables in Rust are immutable by default. The program:

\begin{lstlisting}[label=rmtt,caption=Rust mutability mistake]
fn main() {
    let truth = true;
    truth = false;
}
\end{lstlisting}

will result in a compile error:

\begin{verbatim}
error: re-assignment of immutable variable `truth` [E0384]
\end{verbatim}

To be mutable, the variables must explicitly be annotated with $mut$:

\begin{lstlisting}[label=rmutmtu,caption=Rust mutability]
fn main() {
    let mut lie = true;
    lie = false;
}
\end{lstlisting}

The version I'm using is Rust 1.5.0.

\subsubsection{Object oriented programming}

Rust uses $structs$ to create more complex data types \cite{7_12} \cite{7_13}.
For example, if we want to use two numbers to represent a point in the space:

\begin{lstlisting}[label=rstr,caption=Rust structs]
struct Point {
    x: f32,
    y: f32,
}

fn main() {
    let mut point = Point { x: 0, y: 0 };

    point.x = 3;
    point.y = 4;
    
    println!("The point is at ({}, {})", point.x, point.y);
}
\end{lstlisting}

Rust also has the concept of Traits. A trait is a language feature that tells
the Rust compiler about functionality a type must provide. If we want a method
that tells the point distance to the origin:

\begin{lstlisting}[label=rtrait,caption=Rust traits]
use std::f32;

struct Point {
    x: f32,
    y: f32,
}

impl Point {
    fn distanceFromO(&self) -> f32 {
        f32::sqrt(self.x * self.x + self.y * self.y)
    }
}

fn main() {
    let point = Point { x: 3f32, y: 4f32 }; 
    println!("The distance from the origin is ({})", point.distanceFromO(), );
}
\end{lstlisting}

The Trait implements the $distanceFromO$ method for the $Point$ struct.

Visibility in Rust is done around modules. Rust provides a powerful module 
system that can be used to hierarchically split code in logical units (modules),
and manage visibility (public/private) between them. A module is a collection of
items: functions, structs, traits, impl blocks, and even other modules \cite{7_14}.
\section{Rust language usage}
Rust was made to be a system's programming language. There are lots of ways to
integrate Rust programs into computer systems and other programming languages.
In this usage example we'll see how to write a system shared library in Rust and
use it from other programs in other languages.

\subsection{Our problem}

This is the Monte Carlo algorithm for a Pi approximation 
 \cite{8_1} written in Python:

\begin{lstlisting}[label=pytthth,caption=Python Monte Carlo algorithm]
import random
import sys

def montecarlopi(n):
    m = 0.
    n = int(n)
    for i in range(0, n):
        x, y = random.random(), random.random()
        if ((x ** 2) + (y ** 2)) < 1:
            m = m + 1
    return 4 * m / n

print montecarlopi(sys.argv[1])
\end{lstlisting}

It computes a approximation for the Pi value, based in the given parameter. When
bigger is the parameter, better is the approximation value. Let's measure some
executions:

\begin{verbatim}
$ time python pi.py 10
3.6

real    0m0.016s
user    0m0.015s
sys 0m0.000s
\end{verbatim}

The programs is fast and the value is a very bad approximation. Let's increase
the precision:

\begin{verbatim}
$ time python pi.py 100000
3.14184

real    0m0.067s
user    0m0.058s
sys 0m0.008s

$ time python pi.py 1000000
3.140332

real    0m0.533s
user    0m0.504s
sys 0m0.028s

$ time python pi.py 10000000
3.1423776

real    0m5.192s
user    0m4.912s
sys 0m0.276s

$ time python pi.py 100000000
3.14162228

real    3m55.038s
user    0m52.011s
sys 0m6.175s
\end{verbatim}

The program became very slow very fast. It's a common approach to implement the
slow algorithms of the systems in platforms that runs fast and integrate into
the real workflow by calling this solution in other platform. So, let's take a
look on the implementation of this algorithm in Rust:

\begin{lstlisting}[label=rmcalg,caption=Rust Monte Carlo algorithm]
extern crate rand;

use std::env;
use std::str::FromStr;
use rand::distributions::{IndependentSample, Range};

fn montecarlopi(n: u32) -> f32 {
    let between = Range::new(-1f32, 1.);
    let mut rng = rand::thread_rng();

    let total = n;
    let mut in_circle = 0;

    for _ in 0..total {
        let a = between.ind_sample(&mut rng);
        let b = between.ind_sample(&mut rng);
        if a*a + b*b <= 1. {
            in_circle += 1;
        }
    }
    4. * (in_circle as f32) / (total as f32)
}

fn main() {
   let args: Vec<String> = env::args().collect();
   if args.len() > 1 {
     println!("{}", montecarlopi(FromStr::from_str(&args[1]).unwrap()));
   }
}
\end{lstlisting}

It Is the same algorithm at all. Let's measure the execution of this program:

\begin{verbatim}
$ time ./release/point 100000000
3.1413338

real    0m5.323s
user    0m1.911s
sys 0m3.410s

time ./release/point 1000000000
3.1415544

real    0m53.193s
user    0m19.046s
sys 0m34.107s
\end{verbatim}

Awesome! The execution time is much better (even with more precision). Now we
should make it callable from Python and other platforms.

\subsection{A Rust library}

Let's start creating a new project with Cargo:

\begin{verbatim}
$ cargo new montepy
$ cd montepy
\end{verbatim}

Now, edit the ./src/lib.rs file Cargo creates inserting our previous code with
some changes I'll explain later:

\begin{lstlisting}[label=rlllabrar,caption=Rust library code]
extern crate rand;

use std::env;
use std::str::FromStr;
use rand::distributions::{IndependentSample, Range};

#[no_mangle]
pub extern fn montecarlopi(n: u32) -> f32 {
    let between = Range::new(-1f32, 1.);
    let mut rng = rand::thread_rng();

    let total = n;
    let mut in_circle = 0;

    for _ in 0..total {
        let a = between.ind_sample(&mut rng);
        let b = between.ind_sample(&mut rng);
        if a*a + b*b <= 1. {
            in_circle += 1;
        }
    }
    4. * (in_circle as f32) / (total as f32)
}
\end{lstlisting}

The not so odd modification is the $pub extern$ annotations in the font of
the $montecarlopi$ function. The $pub$ means that this function should be
callable from outside of this module, and the $extern$ says that it should be
able to be called from C. Also the $main$ method was removed because we don't
want this program to execute itself. The other modification was the $\#[no_mangle]$ annotation over the method. When you create a Rust library, it
changes the name of the function in the compiled output. This attribute turns
that behavior off  \cite{8_2}.

Also we had to change the $Cargo.toml$ file. It will look like this:

\begin{verbatim}
[package]
name = "montepy"
version = "0.1.0"
authors = ["André Taiar <taiar@dcc.ufmg.br>"]

[dependencies]
rand = "*"

[lib]
name = "montepi"
crate-type = ["dylib"]
\end{verbatim}

The additions here where the $dependencies$ clause, specifying that rand library
is our dependence (Cargo will download and install it for us), the $lib$ clause,
specifying we are building a library and we want to compile our library into a
standard dynamic library  \cite{8_3}. Let's build
the library:

\begin{verbatim}
$ cargo build --release
    Updating registry `https://github.com/rust-lang/crates.io-index`
   Compiling winapi v0.2.5
   Compiling libc v0.2.2
   Compiling winapi-build v0.1.1
   Compiling advapi32-sys v0.1.2
   Compiling rand v0.3.12
   Compiling montepy v0.1.0 (file:///home/taiar/dev/mono1/rust/montepy)
\end{verbatim}

Now in the $./target/release/$ directory there is a file called
$libmontepi.so$ which is our compiled dynamic library. Now, we can create a
python program that uses this library and compute our result very fast. Here is
the code:

\begin{lstlisting}[label=perrr,caption=Python wrong code for calling the library]
from ctypes import cdll

lib = cdll.LoadLibrary("target/release/libmontepi.so")
print lib.montecarlopi(100000000)

print("done!")
\end{lstlisting}

The execution result:

\begin{verbatim}
$ time python pi.py 
22862944
done!

real    0m5.412s
user    0m2.061s
sys 0m3.343s
\end{verbatim}

The program runs really fast but wait, there is a problem. The aproximation
result is totally wrong.

The problem is that the integers and floats we use in our implementation of the
Rust algorithm has different forms from the types of Python. To get around this,
we must convert our numbers for types that are compatible with our system. Here
is our new Python code:

\begin{lstlisting}[label=pytonright,caption=Python right code]
from ctypes import cdll
import ctypes

lib = cdll.LoadLibrary("target/release/libmontepi.so")

montecarlopi = lib.montecarlopi
montecarlopi.restype = ctypes.c_float
print montecarlopi(100000000)

print("done!")
\end{lstlisting}

\begin{verbatim}
$ time python pi.py 
3.14137887955
done!

real  0m5.400s
user  0m1.952s
sys 0m3.447s
\end{verbatim}

and in Node.js (with the ffi package  \cite{8_4}):

\begin{lstlisting}[label=njs,caption=Node calling our Rust library]
var ffi = require('ffi');

var pipi = ffi.Library('./target/release/libmontepi.so', {
    'montecarlopi': ['float', ['int']]
});

console.log(pipi.montecarlopi(100000000));
\end{lstlisting}

\begin{verbatim}
$ time nodejs pi.js 
3.1415293216705322

real  0m5.466s
user  0m2.038s
sys 0m3.420s
\end{verbatim}

and in C (the $libmontepi.so$ file must be available in your system):

\begin{lstlisting}[label=ccorl,caption=C calling our Rust library]
#include <stdint.h>
#include <stdio.h>

float montecarlopi(const int precision);

int main() {
  printf("%f\n", montecarlopi(100000000));
  return 0;
}
\end{lstlisting}

\begin{verbatim}
$ gcc -o pi pi.c -lmontepi
$ time ./pi 
3.141783

real  0m5.367s
user  0m1.925s
sys 0m3.441s
\end{verbatim}


\bibliography{monografia-andre_taiar}

\end{document}
