\documentclass{abnt}
\usepackage[utf8]{inputenc}
\usepackage[num]{abntcite}
\usepackage{graphicx}
\usepackage{url}
\usepackage{caption}
\usepackage{listings}
\usepackage[brazil]{babel}
\addto\captionsbrazil{%
\def\bibname{References}%
\def\abstract{Abstract}%
}

\lstset{
basicstyle=\small\sffamily,
numbers=left,
numberstyle=\tiny,
frame=tb,
columns=fullflexible,
showstringspaces=false
}

\graphicspath{{imagens/}}

\begin{document}
\autor{André Taiar Marinho Oliveira}

\titulo{A comparative study of concurrency models in Elixir and Rust}

\orientador[Orientador:\\]{Prof. Dr. Fernando Magno Quintão Pereira}

\comentario{Apresentado como requisito da disciplina de Monografia em Sistemas de Informação do DCC/UFMG}

\instituicao{Universidade Federal de Minas Gerais \par Instituto de Ciências
Exatas \par Departamento de Ciência da Computação}

\local{Belo Horizonte} \data{2016/2}

\capa
\folhaderosto

\begin{resumo}
There ...\\

\textbf{Keywords}: programming languages, concurrency, elixir, rust.
\end{resumo}

\sumario
\listoffigures

\chapter{Introdução}

Lorem ipsum dolor sit amet, consectetur adipisicing elit, sed do eiusmod tempor incididunt ut labore et dolore magna aliqua. Ut enim ad minim veniam, quis nostrud exercitation ullamco laboris nisi ut aliquip ex ea commodo consequat. Duis aute irure dolor in reprehenderit in voluptate velit esse cillum dolore eu fugiat nulla pariatur. Excepteur sint occaecat cupidatat non proident, sunt in culpa qui officia deserunt mollit anim id est laborum.

\section{Visão geral do assunto}

\section{Objetivo, justificativa e motivação}

\chapter{Contextualização e trabalhos relacionados}

A Quimioinformática é uma área interdisciplinar que envolve a Química e a Informática e consiste no uso de técnicas computacionais aplicadas a uma gama de problemas no campo da Química. Dentre suas utilizações mais básicas estão o armazenamento, indexação, busca de informações relativas a compostos químicos, representação de átomos, moléculas e reações químicas entre moléculas, bases e bibliotecas de compostos químicos e recuperação de informação nestas bases. A Quimioinformática é sistematicamente utilizada para análise /in silico/ (análises feitas através de simulação computacional) nas indústrias farmacêuticas, na Bioinformática e em qualquer laboratório de Química atualmente.

A Química Combinatória é uma grande ferramenta para a descoberta e o desenvolvimento de novos compostos químicos de interesse em diversas áreas. Nesta metodologia de síntese os produtos são formados simultaneamente através da combinação de diversos compostos e os experimentos são analisados depois de gerados. A principal meta desta metodologia é reduzir o tempo necessário para, por exemplo, a obtenção de um novo fármaco. É fácil perceber que na abordagem combinatorial gasta-se tempo na composição de uma quimioteca (uma biblioteca de experimentos já verificados de forma combinatória) bem como gastam-se compostos químicos que nem sempre serão utilizados em sua totalidade ou mesmo irão gerar um resultado de interesse para a análise final.

Com a popularização e uma maior utilização da computação e da Quimioinformática, é possível reduzir o tempo gasto e a utilização de compostos em experimentos combinatoriais. Utilizando bases pré-definidas de reagentes, preparadas e filtradas arbitrariamente e combinando tais reagentes através de uma reação química conhecida, podemos observar as características dos compostos resultantes e obter uma boa aproximação do que aconteceria com os experimentos na realidade.

Neste trabalho, foi desenvolvido um sistema computacional com interface web que permite a busca por reagentes de interesse em bases de compostos químicos através de diversos filtros de propriedades de moléculas. A partir desta busca, o usuário terá uma visualização dos resultados obtidos e a possibilidade de reagir quimicamente os compostos encontrados na busca, obtendo ao final os dados de todos os compostos possíveis gerados a partir da reação química entre todas as combinações encontradas nas bases filtradas. Estes resultados são automaticamente exportados em um formato de planilha eletrônica, contendo a biblioteca gerada e permitindo salvar a informação e fazer futuras análises com os experimentos gerados.

Em nosso protótipo utilizamos uma base de reagentes da Sigma-Aldrich que é uma grande empresa norte americana do ramo da química, biotecnologia e ciências da vida. Essa empresa é um conceituado fornecedor de compostos químicos para laboratórios de todo o mundo. Seu catálogo é fornecido em diversos formatos pelo ZINC que é uma base de dados (não comerciais) de compostos químicos comercialmente disponíveis, além de uma infinidade de outras funcionalidades ligadas à Bioinformática e farmacêutica. O ZINC é uma ferramenta gratuita desenvolvida e fornecida por laboratórios do Departamento de Química Farmacêutica da Universidade da Califórnia, em São Francisco (UCSF), EUA.

Foram utilizadas muitas ferramentas, todas disponíveis gratuitamente, na composição do sistema desenvolvido. Dentre as que compõem o back-end do projeto, a mais importante é o Open Babel. Segundo palavras do próprio site o Open Babel é um conjunto de ferramentas projetado para lidar com as muitas linguagens dos dados químicos. No sistema ele é responsável por ler as bases de reagentes (disponibilizados em um formato bem específico para moléculas) e buscar os componentes, converter formatos de dados químicos, extrair informações das moléculas e até mesmo gerar a visualização em imagem das moléculas para a web.

Todas as moléculas no sistema são representadas por meio de uma notação chamada SMILES (/Simplified Molecular Input Line Entry System/). Esta notação foi criada por uma empresa chamada Daylight, que também é uma referência na produção de sistemas químicos. Para procurar por determinados padrões e características nas moléculas, utilizamos uma linguagem, também criada pela Daylight, chamada SMARTS. A mesma empresa também fornece uma linguagem para descrição de reações e transformações químicas chamada SMIRKS. Apesar disso, para as reações em nosso sistema não utilizamos os SMIRKS mas sim, uma outra linguagem chamada Reaction SMARTS. Essa linguagem basicamente usa SMARTS para descrever uma transformação química. Para executar a reação química utilizamos uma outra ferramenta livre chamada RDKit.

\chapter{Método do trabalho}

Este trabalho foi definitivamente o meu primeiro contato com a Bioinformática durante toda a minha graduação e o primeiro passo necessário foi tentar entender bem o problema que a minha orientadora e idealizadora do projeto, Rafaela Ferreira, me passou em diversas reuniões que fizemos no início do meu trabalho. Juntamente com a minha outra orientadora, Raquel Minardi, aos poucos as funcionalidades necessárias foram ficando claras para mim e finalmente conseguimos definir um escopo de funcionalidades para este projeto.

Desde as reuniões mais iniciais sobre o tema deste trabalho, mesmo antes de entender corretamente o escopo do projeto, já era determinado que o mesmo deveria ser disponibilizado em forma de um sistema web por diversos motivos: facilidade para a manutenção em um lugar centralizado, facilidade para atualização, manutenção, distribuição e internacionalização do sistema.

Por uma preferência pessoal de forma de trabalho, foi utilizada uma metodologia de trabalho muito parecida com metodologias ágeis de desenvolvimento de software, mais especificamente o Scrum. Houve a primeira parte do trabalho aonde os requisitos do projeto foram ficando claros ao longo de diversas reuniões, em um formato de pré-jogo do Scrum. Nessas reuniões basicamente foram tomadas notas e muito conteúdo era levado para ser pesquisado e analisado. Houve durante esta etapa do projeto um trabalho sobre análise de viabilidade, do que era possível ou não fazer dentro do que foi idealizado inicialmente para o projeto.

Após definido inicialmente o que seria possível fazer dentro do esperado para o sistema, foram definidas algumas funcionalidades prioritárias iniciais em forma de um backlog de projeto e gradativamente este backlog foi sendo desenvolvido e agregado ao projeto principal na forma de Sprints. As Sprints, na maioria das vezes, duraram 1 semana. Sempre iniciavam com o planejamento das funcionalidades a serem implementadas para a próxima iteração do projeto. Em seguida, haveria a fase de desenvolvimento do projeto e, ao final do desenvolvimento, uma reunião de revisão da Sprint. Nesta eram mostradas as funcionalidades que foram desenvolvidas ao decorrer da Sprint para a professora Rafaela (que automaticamente tomou um papel de Product Owner). Então o ciclo se repetia. Ainda sobraram no backlog do produto final entregue neste trabalho, diversas funcionalidades que ainda são bastante interessantes e muito trabalho futuro ainda será possível a partir deste trabalho inicial, como descreverei melhor na conclusão do trabalho.

Acredito que o andamento do trabalho não poderia ter ocorrido de melhor forma, dadas as dificuldades que tive inicialmente para encontrar as minhas orientadoras e definir o meu tema. Trabalhar de acordo com metodologias ágeis permitiu que as funcionalidades prioritárias do sistema fossem desenvolvidas primeiro e que um refinamento progressivo do que deveria ou não entrar em cada fase do desenvolvimento fosse naturalmente clara ao longo do processo. O contato periódico com as orientadoras ajudou bastante a tirar as dúvidas e acelerar o processo de elicitação dos requisitos. Acredito ter escolhido o método correto para trabalhar nessa situação. O atraso inicial do projeto provavelmente valeu a pena dado que eu me interessei realmente pelo tema (foi bastante satisfatório o desenvolvimento do trabalho para mim) e escolhi excelentes orientadoras que me ajudaram bastante no trabalho e a quem muito agradeço pela atenção e disponibilidade.

\chapter{Desenvolvimento do trabalho}

\section{Definição e requisitos do projeto}

O intuito do projeto é desenvolver uma plataforma gratuita e pública para que usuários de laboratórios de todo o mundo possam buscar e experimentar reações químicas entre bases de reagentes moleculares com determinadas características de interesse que possam ser filtradas. Ao final do processo, uma biblioteca com todas as informações geradas deve ser exportada em um formato acessível para que o usuário possa analisar as reações ocorridas e quais compostos são de seu interesse.

Com tais requisitos foi vislumbrado que o ideal seria desenvolver um sistema em plataforma web, disponível através do browser. A base a ser utilizada seria inicialmente a base de compostos oferecidos pela empresa Sigma-Aldrich (pela relevância que tal empresa tem no mercado pela venda de seus reagentes químicos) mas que poderia, futuramente, ser expandido para a busca de reagentes em diversas bases.

Para selecionar as moléculas de interesse existentes nessa grande base de reagentes (na base de compostos da Sigma-Aldrich oferecida pelo ZINC e atualizada em 21/04/2015 existem mais de 100.000 compostos) precisamos filtrar segundo alguns critérios de interesse. O primeiro tipo de critério utilizado para a filtragem nesta base é uma busca por grupos funcionais. Esta filtragem foi implementada por meio de um casamento de padrões utilizando SMARTS dos grupos funcionais. A aplicação já provê alguns SMARTS pré-definidos: Ácido Carboxílico, Amina e Amida, e dá a opção para que o usuário com conhecimento em SMARTS possa fazer o input de seu SMARTS de interesse. Uma das melhorias previstas para fazer futuramente é o input do SMARTS através do desenho da molécula, utilizando alguma ferramenta gráfica que dê suporte a tal funcionalidade.

O segundo tipo de informação que pode ser filtrada dentro das bases de reagentes tem haver com características específicas de cada molécula. Foram implementados campos para input de valores mínimos e máximos esperados para cada um dos critérios filtrados. Os critérios possíveis para filtragem nas moléculas atualmente são: massa molecular, número de átomos, doadores de ligação de hidrogênio, receptores de ligação de hidrogênio, número de anéis e LogP. Todos podem ser combinados com suas quantidades mínimas e máximas esperadas para que o resultado seja refinado arbitrariamente.

É necessário a geração de mais de uma base para que possamos fazer a reação entre bases diferentes. Atualmente o sistema permite gerarmos apenas duas bases simultaneamente. Uma das funcionalidades previstas para as próximas versões é permitir a geração de mais de duas bases através de filtragem e poder reagir todas as bases, gerando assim ainda mais resultados.

Após a busca nas bases, os resultados dos reagentes encontrados são exibidos. Em uma tabela são listados os desenhos das moléculas, sua fórmula e todos os critérios de filtragem citados anteriormente com o valor pertencente a cada uma das moléculas. Com tais bases geradas, é possível executar a simulação da reação química entre todos os reagentes pertencentes a cada uma delas. Para isso, basta selecionar a reação desejada em uma lista de reações já disponíveis na aplicação.

Para executar as reações, utilizamos uma linguagem chamada Reaction SMARTS. Ao contrário dos SMARTS de busca de padrões nas moléculas que usamos anteriormente, ainda não disponibilizamos o input de Reaction SMARTS na aplicação pelo usuário (mas isso está previsto para acontecer posteriormente). Após a seleção da reação e o comando para executá-la, é gerado um arquivo em formato de planilha eletrônica e automaticamente é feito o seu download.

Mais resultados e exemplos da utilização do sistema serão vistos no capítulo Resultados e discussão.

\section{Verificação da viabilidade}

Com os requisitos da aplicação em mãos, precisamos buscar os componentes certos para compor o trabalho de forma a viabilizar que tudo funcione como planejado. O primeiro problema resolvido foi a forma de pesquisa na base de dados. O ZINC fornece as bases que iremos trabalhar em diversos formatos de dados. Os que nos interessam são os formatos MOL2 e PDB. Estes formatos contém diversas informações sobre as moléculas já pré-calculadas que serão úteis para a extração de outras informações necessárias no projeto. O Open Babel é a ferramenta que utilizamos para trabalhar com estas bases de dados. Por ser uma ferramenta especializada nesse tipo de operações (trabalhar com dados de bases de dados químicas, conversão dos formatos de arquivos, busca etc) acabou se tornando o principal componente do back-end em nossa aplicação.

O segundo problema a ser resolvido era a execução das reações. Existem atualmente boas ferramentas comerciais que realizam reações /in silico/ de moléculas mas são poucas as opções gratuitas que fazem um trabalho satisfatório. A única opção encontrada e adotada no trabalho até o momento é através do RDKit. Como seu próprio slogan diz, ele é um programa open source de quimioinformática. O RDKit é uma suíte grande e completa e com muitas funcionalidades para desenvolver software químico. Neste trabalho, apenas adaptamos um programa já existente que utiliza o RDKit para executar as reações e foi fornecido pelo ZINC.

Após resolver estes problemas, tivemos segurança de que o projeto era possível de ser executado como queríamos e partimos para a implementação da plataforma.

\section{Definição da tecnologia}

Foi escolhido para compor a aplicação a linguagem Ruby com o framework Ruby on Rails por diversos motivos. Primeiramente, pela facilidade que este traz para escrever aplicações web de forma rápida e prática. Em segundo lugar, o Open Babel disponibiliza toda a sua API também para a linguagem Ruby, o que seria essencial para utilizá-lo como engine de busca e conversões de formatos dentro da aplicação. O RDKit infelizmente não oferece bindings para linguagem Ruby como o Open Babel, ele oferece suporte apenas à linguagem Python. Nesse caso, tivemos que trabalhar com chamadas de sistema diretamente à aplicação que utiliza o RDKit.

\section{Descrição da arquitetura do projeto}

Este é um diagrama da arquitetura do projeto com os principais componentes do nosso sistema:

\bibliography{monografia-andre_taiar}

\end{document}
