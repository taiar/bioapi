\documentclass{abnt}
\usepackage[utf8]{inputenc}
\usepackage[num]{abntcite}
\usepackage{graphicx}
\usepackage{url}
\usepackage{caption}
\usepackage{listings}
\usepackage[brazil]{babel}
\addto\captionsbrazil{%
\def\bibname{References}%
\def\abstract{Abstract}%
}

\lstset{
basicstyle=\small\sffamily,
numbers=left,
numberstyle=\tiny,
frame=tb,
columns=fullflexible,
showstringspaces=false
}

\graphicspath{{imagens/}}

\begin{document}
\autor{André Taiar Marinho Oliveira}

\titulo{A comparative study of concurrency models in Elixir and Rust}

\orientador[Orientador:\\]{Prof. Dr. Fernando Magno Quintão Pereira}

\comentario{Apresentado como requisito da disciplina de Monografia em Sistemas de Informação do DCC/UFMG}

\instituicao{Universidade Federal de Minas Gerais \par Instituto de Ciências
Exatas \par Departamento de Ciência da Computação}

\local{Belo Horizonte} \data{2016/2}

\capa
\folhaderosto

\begin{resumo}
There are hundred of programming languages out there. Which one should we use?
Which help do we have to choose well? How do they compare to each other? This
document is an attempt to provide some answers to these questions. Naturally, it
would not be possible to provide complete answers: as I mentioned, there are too
many programming languages. Nevertheless, we chose four languages with a
potential to grow in importance in the coming years. These programming languages
are Ceylon, Dart, Elixir and Rust. During this project, we shall be
talking about each one of them. These discussions will be in breath, not in
depth. Their goal is to provide the reader with the minimum of information
necessary to compare them, and who knows, to lure one or other interested person
in learning them in a greater level of details. In any case, we hope to
contribute a bit to the popularization of these programming languages, which -
likely - will be paramount to the development of computer science in the next
ten years.\\

\textbf{Keywords}: programming languages, concurrency, elixir, rust.
\end{resumo}

\sumario
\listoffigures

\chapter{Elixir}
\section{Elixir language introduction}
\input{elixir_intro}
\section{Elixir language usage}
\input{elixir_usage}

\chapter{Rust}
\section{Rust language introduction}
\input{rust_intro}
\section{Rust language usage}
\input{rust_usage}

\bibliography{monografia-andre_taiar}

\end{document}
