\subsection{How did it appear?}

Rust  \cite{7_1} is a general-purpose,
multi-paradigm, compiled, systems programming
language. Nowadays, the Rust project is maintained and developed by Mozilla  \cite{7_2}
Research but it starts as a personal project by Mozilla employee Graydon Hoare
 \cite{7_3}  \cite{7_4}.

Graydon starts working on Rust in 2006. He was already a engineer at Mozilla
working on compilers and tools for other languages. He started sketching Rust
thinking in a lot of obvious good ideas, known and loved in other languages,
that haven't made it into widely-used systems languages, or are deployed in
languages that have very poor (unsafe, concurrency-hostile) memory models.

After working long time as a hobby project, he decided to show one such
prototype he'd been working on his spare time to his manager. Mozilla took an
interest and set up a team to work on this language as a component of
longer-term project to rebuild their browser stack around safer, more
concurrent, easier technologies than C++. That larger project is called \"servo\".
Mozilla is funding Rust development because of that.

Mozilla began sponsoring the project in 2009 and announced it in 2010. The
first Rust compiler, that was written in OCaml 
 \cite{7_5}, was dropped and rewritten in a
Rust version that successfully compiled itself in 2011. It is called $rustc$
and uses LLVM  \cite{7_6} as it's back end. The
version 1.0  \cite{7_7} was released in 15 May
2015.

\subsection{Why was it designed and implemented?}

System's programming languages became widely used on the last 50 years since
people started using high-level languages to write operating systems.
Nevertheless, there were always two problems constantly present in this niche:

\begin{itemize}
    \item It's difficult to write secure code (due to the way that languages like C and
          C++ manages memory);
    \item It's hard (but essential) to write multi-threaded concurrent code due to the
          low support the languages of this niche offers to programmers to deal
          with the bugs that writing parallel programs would offer  \cite{7_8}.
\end{itemize}

These are the problems that Rust was made to address. As it's own creator says:

\begin{quotation}
Our target audience is frustrated C++ developers. 
\end{quotation}

on which he includes himself. The performance of safe code (which is one of
things that Rust aims) is expected to be slower than C++, however, the Rust
program's performance is comparable to C++ code that manually implements safer
operations  \cite{7_9}. 

Rust is a compiled, multi-paradigm programming language. It supports
pure-functional, actor-based concurrency, imperative-procedural and
object-oriented programming styles.

\subsection{How can we use it, e.g., install and run?}

\subsubsection{Linux and Mac OS}

Rust has a simple installation proccess on Linux and Mac OS systems. Run the
following line in the command terminal:

\begin{verbatim}
$ curl -sSf https://static.rust-lang.org/rustup.sh | sh
\end{verbatim}

It will download the files, ask for $sudo$ passwords and setup everything
automatically.

\subsubsection{Windows}

There are binaries versions with installers for Windows users at the download
page  \cite{7_10}.

\subsubsection{Testing the installation}

Rust default toolset ships with a build tool called $cargo$
 \cite{7_11}. It is a tool to help manage the
Rust projects. Cargo manages three things: building your code, downloading the
dependencies your code needs, and building those dependencies.

To start a new project with cargo, run the command:

\begin{verbatim}
$ cargo new hello --bin
\end{verbatim}

It will generate the dicrectory called $hello$ with some files on it. On the
directory root it creates a $Cargo.toml$ file with some basic information about
the project and a $src$ directory with a $main.rs$ file on it. This is the
content of the main file:

\begin{lstlisting}[label=rhw,caption=Rust Hello World]
fn main() {
    println!("Hello, world!");
}
\end{lstlisting}

To compile and run the project, in the project's root directory, run the
commands:

\begin{verbatim}
$ cargo build
   Compiling hello v0.1.0 (file:///home/taiar/dev/mono1/rust/hello)
$ cargo run
     Running `target/debug/hello`
Hello, world!
\end{verbatim}

and the hello message just show up. The binary file is located at
$target/debug/hello$ and can be executed directly from the command line
without the cargo tool too.

\begin{verbatim}
$ ./target/debug/hello
Hello, world!
\end{verbatim}

\subsection{Simple programs}

\subsubsection{Primitive types and mutability}

Rust provides access to a wide variety of primitives:

\begin{itemize}
    \item signed integers: $i8$, $i16$, $i32$, $i64$ and $isize$ (pointer size);
    \item unsigned integers: $u8$, $u16$, $u32$, $u64$ and $usize$ (pointer size);
    \item floating point: $f32$, $f64$;
    \item char Unicode scalar values like $'a'$, $\alpha$ and $\infty$ (4 bytes each);
    \item bool either $true$ or $false$;
    \item arrays like $[1, 2, 3]$;
    \item tuples like $(1, true)$.
\end{itemize}

Here are some usage:

\begin{lstlisting}[label=rst,caption=Rust types usage]
fn main() {
    // Variables can be type annotated.
    let logical: bool = true;

    let a_float: f64 = 1.0;  // Regular annotation
    let an_integer   = 5i32; // Suffix annotation

    // Or a default will be used.
    let default_float   = 3.0; // `f64`
    let default_integer = 7;   // `i32`

    let mut mutable = 12; // Mutable `i32`.
}
\end{lstlisting}

Variables in Rust are immutable by default. The program:

\begin{lstlisting}[label=rmtt,caption=Rust mutability mistake]
fn main() {
    let truth = true;
    truth = false;
}
\end{lstlisting}

will result in a compile error:

\begin{verbatim}
error: re-assignment of immutable variable `truth` [E0384]
\end{verbatim}

To be mutable, the variables must explicitly be annotated with $mut$:

\begin{lstlisting}[label=rmutmtu,caption=Rust mutability]
fn main() {
    let mut lie = true;
    lie = false;
}
\end{lstlisting}

The version I'm using is Rust 1.5.0.

\subsubsection{Object oriented programming}

Rust uses $structs$ to create more complex data types \cite{7_12} \cite{7_13}.
For example, if we want to use two numbers to represent a point in the space:

\begin{lstlisting}[label=rstr,caption=Rust structs]
struct Point {
    x: f32,
    y: f32,
}

fn main() {
    let mut point = Point { x: 0, y: 0 };

    point.x = 3;
    point.y = 4;
    
    println!("The point is at ({}, {})", point.x, point.y);
}
\end{lstlisting}

Rust also has the concept of Traits. A trait is a language feature that tells
the Rust compiler about functionality a type must provide. If we want a method
that tells the point distance to the origin:

\begin{lstlisting}[label=rtrait,caption=Rust traits]
use std::f32;

struct Point {
    x: f32,
    y: f32,
}

impl Point {
    fn distanceFromO(&self) -> f32 {
        f32::sqrt(self.x * self.x + self.y * self.y)
    }
}

fn main() {
    let point = Point { x: 3f32, y: 4f32 }; 
    println!("The distance from the origin is ({})", point.distanceFromO(), );
}
\end{lstlisting}

The Trait implements the $distanceFromO$ method for the $Point$ struct.

Visibility in Rust is done around modules. Rust provides a powerful module 
system that can be used to hierarchically split code in logical units (modules),
and manage visibility (public/private) between them. A module is a collection of
items: functions, structs, traits, impl blocks, and even other modules \cite{7_14}.